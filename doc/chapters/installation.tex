\chapter{Installation and Configuration}

\section{Software requirements}

The \fmipp \matlab FMU Export Utility is intended to run on Windows~7~(32-bit).
A working \matlab installation is required.
The toolbox has been tested with \matlab~R2013a.


When exporting \matlab functionality as an FMU for Co-Simulation, a working \matlab installation is also required on the system where the FMU is used (with a valid license for all the \matlab toolboxes that are used by the FMU).
Furthermore, for the export of FMUs the following tools need to be installed:
\begin{itemize}

  \item \href{https://www.microsoft.com/en-us/download/details.aspx?id=44914}{Microsoft Visual Studio Express 2013}

  \item \href{https://www.python.org/}{Python~2}
  
\end{itemize}


\section{Installation}
\label{sec:install}

To install the \fmipp \matlab Toolbox proceed as follows:
\begin{itemize}
  \item Download the latest version of the toolbox as ZIP-file from the \href{https://sourceforge.net/projects/matlab-fmu/files/latest/download}{download page}.
  
  \item Unzip the installation file into any sub directory (referred to as the \emph{installation folder}).
  
  \item Create a new Windows environment variable called \texttt{MATLAB\_FMIPP\_ROOT} that points to the installation folder.
  See for instance \href{http://www.computerhope.com/issues/ch000549.htm}{here} for instructions how to do that.
  
  \item Whenever starting \matlab, run the script \texttt{setup.m} in the installation folder.
\end{itemize}

For instance, after unzipping the toolbox to \texttt{C:\\Program~Files\\matlab-fmipp}, specify this path as the value of environment variable \texttt{MATLAB\_FMIPP\_ROOT}.
Whenever you start \matlab, change to the installation folder:
\begin{verbatim}
  >> cd 'C:\Program Files\matlab-fmipp'
\end{verbatim}
Then run the setup script by typing:
\begin{verbatim}
  >> setup
\end{verbatim}