\chapter{Importing FMUs into \matlab}


\section{About}

The \href{https://fmi-standard.org/}{Functional Mock-up Interface}~(FMI) specification intentionally provides only the most essential and fundamental functionalities in the form of a C interface.
On the one hand, this increases flexibility in use and portability to virtually any platform.
On the other hand, such a low-level approach implies several prerequisites a simulation tool has to fulfil in order to be able to utilize such an FMI component.

The \fmipp \matlab Toolbox provides a wrapper for the \href{http://fmipp.sourceforge.net}{\fmipp Library}, which intends to bridge the gap between the basic functionality provided by the FMI specification and the typical requirements of simulation tools.
The \fmipp Library provides high-level functionality that eases the handling and manipulation of FMUs, such as numerical integration, advanced event-handling or state predictions.
This allows FMUs to be used more easily, e.g., integrating it into fixed time step or discrete event simulations.

This toolbox provides a stand-alone version of the \matlab interface for the \href{http://fmipp.sourceforge.net}{\fmipp Library} for Windows.
For other operating systems, this package can be built from \href{http://sourceforge.net/p/fmipp/code/ci/master/tree/}{source}.


\section{Example Usage}

The \emph{fmippim} package provides classes that allow to manipulate FMUs for ModelExchange and for Co-Simulation.
In the following, short descriptions and code snippets of the provided functionality demonstrate their usage.
More extensive background information can be found in the documentation of the \href{http://fmipp.sourceforge.net}{\fmipp Library}


\subsection{Loading the library}

FMUs are basically ZIP archives.
Before they can be used, they have to be extracted (unzipped), for instance with the help of \matlab's \texttt{unzip} function.
To load the library type:

\begin{verbatim}
  fmippPath = getenv( 'MATLAB_FMIPP_ROOT' );
  addpath( genpath( fullfile( fmippPath, 'packages' ) ) );
\end{verbatim}

\subsection{Classes FMUModelExchangeV1 and FMUModelExchangeV2}

The most obvious obstacle for using a bare FMU for ModelExchange is its lack of an integrator.
For this reason, classes \texttt{FMUModelExchangeV1} and \texttt{FMUModelExchangeV2} provide generic methods for the integration of FMUs for ModelExchange for FMI Version 1.0 and 2.0, respectively.
Instances of these classes own the actual FMU instance and are able to advance the current state up to a specified point in time, including the proper handling of FMU-internal events.
The classes also provide functionality for convenient input and output handling.

The following example demonstrates the basic usage of class \texttt{FMUModelExchangeV1} (usage of class \texttt{FMUModelExchangeV2} is analogous).
The example is also available as a \matlab script in folder \emph{test/import\_me}, together with a Modelica model that can be translated to an FMU for ModelExchange for testing it.

First, specify the FMU's model name and the path to the unzipped FMU (as URI):
\begin{verbatim}
model_name = 'StandaloneRadiator';
uri_to_extracted_fmu = 'file:///C:/path/to/unzipped/StandaloneRadiator';
\end{verbatim}
Then, specify the FMU's configuration parameters and load the FMU:
\begin{verbatim}
logging_on = fmippim.fmiTrue();
stop_before_event = fmippim.fmiTrue();
event_search_precision = 1e-2;
integrator_type = fmippim.bdf(); % CVODE solver (Backward Differentiation Formula).

fmu = fmippim.FMUModelExchangeV1( uri_to_extracted_fmu,
                                  model_name,
                                  logging_on,
                                  stop_before_event,
                                  event_search_precision,
                                  integrator_type )
\end{verbatim}
Instantiate the FMU:
\begin{verbatim}
status = fmu.instantiate( 'standalone_radiator1' )
if status ~= fmippim.fmiOK(); error( 'instantiation not successful' ); end
\end{verbatim}
Set value of parameters:
\begin{verbatim}
Tlow = 82.0;
status = fmu.setRealValue( 'Tlow', Tlow )
if status ~= fmippim.fmiOK(); error( 'setRealValue not successful' ); end
\end{verbatim}
Initialize the FMU.
\begin{verbatim}
status = fmu.initialize()
if status ~= fmippim.fmiOK(); error( 'initialzation not successful' ); end
\end{verbatim}
Specify default step size of one integration step and the internal integrator step size, then start the simulation loop. The integrator always tries to make a full step, but it stops in case an event is detected.
\begin{verbatim}
stepsize = 300;
integrator_stepsize = stepsize/10;

t = 0;
tstop = 4 * 60 * 60;

while t < tstop
    t = fmu.integrate( t + stepsize, integrator_stepsize );

    T = fmu.getRealValue( 'T' );
    derT = fmu.getRealValue( 'derT' );

    if ( ( abs( T - Thigh ) < 1e-2 ) && ( derT > 0 ) )
        fmu.setRealValue( 'Pheat', 0.0 );  % turn off heating
    elseif ( ( abs( T - Tlow ) < 1e-2 ) && ( derT < 0 ) )
        fmu.setRealValue( 'Pheat', 1e3 );  % turn on heating
    end
end
\end{verbatim}
	
The integration algorithms provided by ODEINT and SUNDIALS can be chosen with an appropriate flag in the constructor (see example above).
The following algorithms are available:

\begin{center}

\begin{tabular}{|l|l|l|l|l|}
\hline 
Stepper & Name & Suite & Order & Adaptive \\ 
\hline 
eu & Explicit Euler & ODEINT & 1 & No  \\ 
\hline 
rk & 4th order Runge-Kutta & ODEINT & 4 & No  \\ 
\hline 
abm & Adams-Bashforth-Moulton & ODEINT & 8 & No  \\ 
\hline 
ck & Cash-Karp & ODEINT & 5 & Yes  \\ 
\hline 
dp & Dormand-Prince & ODEINT & 5 & Yes  \\ 
\hline 
fe & Fehlberg & ODEINT & 8 & Yes  \\ 
\hline 
bs & Bulirsch Stoer & ODEINT & 1-16 & Yes  \\ 
\hline 
ro & Rosenbrock & ODEINT & 4 & Yes  \\ 
\hline 
bdf & Backward Differentiation Formula & SUNDIALS & 1-5 & Yes  \\ 
\hline 
abm2 & Adams-Bashforth-Moulton & SUNDIALS & 1-12 & Yes \\ 
\hline 
\end{tabular} 

\end{center}


\subsection{Class FMUCoSimulation}

An example anlogous to the one above but using an FMU for Co-Simulation is available as a \matlab script in folder \emph{test/import\_cs}.
The folder also contains a Modelica model that can be translated to an FMU for Co-Simulation for testing it.


